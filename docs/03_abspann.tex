
\ihead{LITERATUR UND QUELLEN}
\renewcommand{\refname}{Literatur und Quellen}
\addcontentsline{toc}{section}{Literatur und Quellen}
\setlength{\parskip}{.4cm}


\begin{thebibliography}{999999999} 

    \subsection*{Literatur}
    \label{subsec: literatur}

        \bibitem[Gün14]{gün14} \textsc{Günther}, Karsten: \textit{Bash kurz \& gut}. O'Reilly Verlag, Köln. 3. Auflage, 2014, ISBN: 978-3-95561-764-6.

        \bibitem[MR19]{mr19} \textsc{Mandl}, Prof. Dr. Peter und \textsc{Rottmüller}, M. Sc. Björn: \textit{Grundlagen der Bash-Programmierung -- Begleitmaterial zu den Übungen im Kurs Wirtschaftsinformatik II}, Version 4.0.0 online unter

        \url{https://www.wirtschaftsinformatik-muenchen.de/mitarbeiter/peter-mandl/lehrveranstaltungen/wirtschaftsinformatik/}

        (letzter Download am 19. März 2019)

        \bibitem[WK13]{wk13} \textsc{Wolf}, Jürgen und \textsc{Kania}, Stefan: \textit{Shell-Programmierung -- Das umfassende Handbuch}. Galileo Press Computing, Bonn. 4., aktualisierte und erweiterte Auflage, 2013, ISBN: 978-3-8362-2310-2. 

    Das eBook ist in zweiter Auflage kostenfrei erhältlich unter:

        \url{http://linuxint.com/DOCS/Linux_Docs/openbook_shell/shell_001_000.htm#Xxx999207}

    \subsection*{Webseiten}

        \bibitem[UbuUs]{ubuus} \textsc{Ubuntuusers}, online unter \url{https://wiki.ubuntuusers.de/Startseite/} \\ (letzter Abruf am **. April 2019).

        \bibitem[BaWi]{bawi} \textsc{Bash, Wikipedia} unter \url{https://de.wikipedia.org/wiki/Bash_(Shell)} \\ (abgerufen am 26. März 2019).


    \subsection*{Sonstige}

        \bibitem[Sch16]{sch16} \textsc{Schürmann}, Tim: \textit{Shell-Programmierung lernen -- Erste Schritte bei der Automatisierung unter Linux/Unix.}. Video2Brain -- LinkedIn company, Graz, Österreich, 2016. Mehr Infos unter

        \url{https://de.linkedin.com/learning/shell-programmierung-lernen}

        (letzter Abruf am 30. März 2019).

\end{thebibliography}
